\documentclass[12pt,a4paper]{scrartcl}

\usepackage[a4paper, left=2cm, right=2cm, bottom=1cm, top=1cm, includeheadfoot]{geometry}
\usepackage[ngerman]{babel}
%\usepackage[utf8]{inputenc} % comment this if you uncomment utf8x
\usepackage[utf8x]{inputenc} % uncomment this if there are problems with 'ä', 'ü', 'ö'
\usepackage{ucs}
\usepackage[usenames,dvipsnames]{xcolor}
\usepackage[fleqn]{amsmath}
\usepackage{amsfonts}
\usepackage{amssymb}
\usepackage{color}
\usepackage{listings}
\usepackage{hyperref}
\usepackage{amsfonts}
\usepackage{scrpage2}
\usepackage{graphicx}
\usepackage{scrpage2} \pagestyle{scrheadings}
\usepackage[absolute]{textpos}
\usepackage{pdfpages}
\usepackage{alltt}
\usepackage{float}
\usepackage{array}
\usepackage{lastpage}
\usepackage{titling}

% command to format a collum in a table centered
\newcolumntype{C}[1]{>{\centering\arraybackslash}m{#1}}

\definecolor{mygray}{rgb}{0.95,0.95,0.95}
\lstset{language=[Visual]Basic, morekeywords={param, local}}

\lstset{
   literate={ö}{{\"o}}1
           {ä}{{\"a}}1
           {ü}{{\"u}}1
           {ß}{{\ss}}1
           {é}{{\'e}}1,
   inputencoding=ansinew,
   extendedchars=true,
   basicstyle=\scriptsize\ttfamily,
   numberstyle=\scriptsize,
   breaklines=true,
   tabsize=2,
   numbersep=5pt
}

\lstdefinestyle{ccode}{
   language=C,
   keywordstyle=\color{blue}\bfseries,
   stringstyle=\color{BrickRed}\ttfamily,
   commentstyle=\color{OliveGreen}\ttfamily,
   showspaces=false,
   showstringspaces=false,
   showtabs=false,
   numbers=left,
   frame=single
}
\lstdefinestyle{cplusplus}{
   language=C++,
   keywordstyle=\color{blue}\bfseries,
   stringstyle=\color{BrickRed}\ttfamily,
   commentstyle=\color{OliveGreen}\ttfamily,
   showspaces=false,
   showstringspaces=false,
   showtabs=false,
   numbers=left,
   frame=single
}
\lstdefinestyle{csharp}{
   language=[sharp]c,
   keywordstyle=\color{blue}\bfseries,
   stringstyle=\color{BrickRed}\ttfamily,
   commentstyle=\color{OliveGreen}\ttfamily,
   showspaces=false,
   showstringspaces=false,
   showtabs=false,
   numbers=none,
   frame=single
}
\lstdefinestyle{vhdlcode}{
   language=VHDL,
   keywordstyle=\color{blue}\bfseries,
   stringstyle=\color{BrickRed}\ttfamily,
   commentstyle=\color{OliveGreen}\ttfamily,
   showspaces=false,
   showstringspaces=false,
   showtabs=false,
   numbers=left,
   frame=single
}
\lstdefinestyle{customoutput}{
   backgroundcolor=\color{mygray},
   numbers=none,
   showspaces=false,
   showtabs=false
}

\newcommand{\includeccode}[1]{\lstinputlisting[style=ccode]{#1}}
\newcommand{\includecpluspluscode}[1]{\lstinputlisting[style=cplusplus]{#1}}
\newcommand{\includecsharpcode}[1]{\lstinputlisting[style=csharp]{#1}}
\newcommand{\includevhdlcode}[1]{\lstinputlisting[style=vhdlcode]{#1}}
\newcommand{\includetextfile}[1]{\lstinputlisting[style=customoutput]{#1}}

%pictures
\newcommand{\includepicture}[4]{
\begin{figure}[H]
\centering
\includegraphics[scale=#2]{#1}
\caption{\textit{#4}}
\label{fig:#3}
\end{figure}
}

%~~~~~~~~~~~~~~~~~~~~~~~~~~~~~~~~~~~~~~~~~~~~
% Some constants for header, footer and tile page
%~~~~~~~~~~~~~~~~~~~~~~~~~~~~~~~~~~~~~~~~~~~~

% Format der Kopf-und Fußzeile
\pagestyle{scrheadings}
\clearscrheadfoot

\ihead{\emph{FH Hagenberg / ESD}}
\chead{\emph{On Chip Signal Processing}}
\ohead{\emph{RTL Implementierung RX}}

\ofoot{\emph{\thepage \ / \ \pageref{LastPage}}}

\begin{document}  % Beginn des Textes

\pretitle{%
  \begin{center}
  \LARGE
  \includegraphics[scale=0.1]{../img/logo}\\[\bigskipamount]
}
\posttitle{\end{center}}

\title{On Chip Signal Processing - RTL Implementierung RX}
\subtitle{Sommersemester 2018}
\author{Daniel Ferrari \\ Nikolaus Haminger \\ Florian Kitzer \\ Matthäus Kücher \\ Alexander Traxler}

% Insert name and etc. into pdf
%\begin{textblock*}{60mm}(45mm,58mm)
%\firstname \ \lastname , \sfirstname \ \slastname
%\end{textblock*}

\maketitle % Titelseite erstellen
%\includepdf[pages=-]{../spv_ue05.pdf}
\tableofcontents

\newpage

\section{Übersicht}
\subsection{Spezifikation}
In diesem Projekt sollte eine OFDM-RX-Architektur in VHDL implementiert werden. Ein grobes Blockdiagramm dieser RX-Komponente war vorgegeben und ist in Abbildung \ref{fig:big_picture} sichtbar.

\includepicture{../img/big_picture.png}{0.45}{big_picture}{Blockschaltbild der gesamten RX-Kette.}

\subsection{Implementierung}
Die Implementierung wurde weitgehend nach dem Vorbild des spezifizierten Blockschaltbildes vollzogen. Allerdings wird der Offset und das Delay in der Entity \emph{Coarse Alignment} gehandhabt, weshalb die Leitungen \texttt{rx\_data\_delay}, \texttt{rx\_data\_offset} und \texttt{interp\_mode} zwischen \emph{Interpolation} und \emph{Coarse Alignment} eliminiert wurden.

\subsection{Toplevel-Simulation}
Zur Verifikation der gesamten RX-Kette wurde eine Toplevel-Simulation erstellt. Für den Vergleich der Ergebnisse der Hardware-Kette wurde ein Golden Model in Python implementiert. Diese Implementierung berechnet die Input-Signale für die Hardware-Simulation (\texttt{rx\_data\_i}), speichert diese in CSV-Dateien ab und simuliert die gesamte RX-Kette in einer Hardware-ähnlichen Implementierung. Die Ergebnisse des Golden Models werden ebenfalls abgespeichert und am Ende der Hardware-Simulation mit eben jenen Ergebnissen verglichen. Außerdem werden \emph{BER} und \emph{EVM} berechnet sowie Konstellationdiagramme jeweils für beide Simulationen erstellt.

\vspace{10pt}
\noindent
\textbf{Achtung}: Matlab wird nicht benötigt, da das gesamte Golden Model sowie die Verifikation in Python implementiert wurde (mit Fixed-Point-Repräsentation und Taylor-Approximation der Interpolation). Die Gründe für diese Maßnahme sind die bessere Performance von Python, die einfachere Interaktion aus der Modelsim-Simulation und die bessere Strukturierungsmöglichkeit des Codes.

\subsubsection{Anleitung zur Toplevel-Simulation}
Eine ausführliche Anleitung befinden sich in der \texttt{README.md} des Repositories. Hier kurz noch einmal zusammengefasst:

\vspace{5pt}
\noindent
\textbf{Anforderungen}:
\begin{itemize}
\item Modelsim oder Questasim
\item Python3 (https://www.python.org/downloads/)
\item numpy und matplotlib (\texttt{pip install numpy matplotlib})
\end{itemize}

\vspace{5pt}
\noindent
\textbf{Ausführung}:\\
Um die Simulation zu starten, einfach \texttt{start\_rx\_simulation.bat <cmd|gui|clean>} oder\\ \texttt{start\_rx\_simulation.sh <cmd|gui|clean>} im Root-Folder des Repositories ausführen.

\subsubsection{Simulations-Output}
Nachfolgend ist der Output einer Simulation von 3 unterschiedlichen RX-Sequenzen zu sehen.
\includetextfile{sim_output.txt}

\subsubsection{Konstellationsdiagramm}
Abbildung \ref{fig:plot0} zeigt die Konstellationsdiagramme einer Symbol-Sequenz für die Golden-Model- und die Hardware-Simulation. Die Plots sind nach einer Simulation im Ordner\\ \texttt{src/grpRx/unitTopLevel/sim} zu finden.
\includepicture{../img/scatter_plot0.png}{0.65}{plot0}{Konstellationsdiagramme der Golden-Model- und Hardware-Simulationen}

\section{Komponenten}

\subsection{Interpolation}

\subsection{Coarse Alignment}
Die Aufgabe des \texttt{Coarse Alignment} ist das Finden des Beginns einer OFDM-Übertragung. Für das Finden der Beginns wird die Methode nach Schmidl und Cox verwendet. Dazu wird in Hardware die Korrelation über zwei identische Halbsymbole (= Trainingssymbol) gerechnet und das Maximum detektiert. Danach werden fortlaufend die Steuersignale \texttt{start\_of\_symbol} und \texttt{data\_valid} des Datenstromes gesetzt und nach den Vorgaben des \texttt{Fine Alignment} korrigiert.

\subsubsection{Architektur}

\begin{center}
\includepicture{../img/arch_coarse_alignment.png}{0.70}{arch_coarse}{Architektur des \texttt{Coarse Alignment}-Blocks. Links oben sind die Eingangsdaten als I und Q Anteil zu sehen. Darauf folgt die Korrelationsberechnung, gefolgt von der FSM zum Erkennen des Maximums. Unten im Bild ist die Logik zur Steuersignalgenerierung zu sehen.}
\end{center}

\subsubsection{Simulation}
Für die Simulation wurde sowohl Matlab als auch Modelsim verwendet. Mit Hilfe von Matlab wurde ein Empfangssignal mit der Länge von 100 Symbolen generiert. In diesem Empfangssignal ist das Trainingssymbol an der zweiten Stelle. Der Beginn der Korrelation des generierten Empfangssignals ist in Abbildung \ref{fig:signal_coarse} zu sehen. Der grün umrandete Signalteil stellt das Trainingssymbol dar. Das obere Diagramm zeigt das Ergebnis der iterativen Implementierung. Das untere Diagramm zeigt das Ergebnis der formalen Implementierung. Es ist zu erkennen, dass beide Diagramme bis auf eine Verschiebung das gleiche Ergebnis darstellen.

\begin{center}
\includepicture{../img/p_signal.png}{0.33}{signal_coarse}{Beginn des Empfangssignals. Oben ist die iterative und unten die formale Implementierung dargestellt. Der grün umrandete Signalteil stellt das Trainingssymbol dar.}
\end{center}

Für die Simulation wurde der Simulator Modelsim verwendet. Als Testframework wurde UVVM verwendet. Dadurch war es möglich ein besser strukturierte VHDL-Testbench zu entwerfen. In der Simulation wird das mit Matlab generierte Empfangssignal an den \texttt{Coarse Alignment}-Block angelegt und auf die Detektion des Trainingssymbols gewartet. Nach der Detektion wird das Timing der Steuersignale überprüft. Darauf wird überprüft, ob die Daten korrekt vom Block weiter gegeben werden. Am Ende wird das Verhalten im Zusammenhang mit den Steuerleitungen vom \texttt{Fine Alignment} und dem \texttt{init}-Signal überprüft. In Abbildung \ref{fig:sim_coarse} ist ein Teil der Waveform der Simulation zu sehen. Die berechnete Korrelation in der Simulation stimmt mit der Berechnung in Matlab überein.

\begin{center}
\includepicture{../img/sim_coarse.png}{0.30}{sim_coarse}{Waveform der Simulation des \texttt{Coarse Alignment}. Im unteren Teil der Bildes ist die Korrelation des Empfangssignales zu erkennen.}
\end{center}

Um die Simulation starten zu können, müssen die Skripen \texttt{compile\_uvvm.do}, \texttt{compile.do} und \texttt{run\_simulation.do} in dieser Reihenfolge ausgeführt werden.

\subsubsection{Synthese}
Für den implementierten \texttt{Coarse Alignment}-Block wurde eine Synthese mit Quartus durchgeführt. Als Zielplattform wurde FPGA 5CSXFC6D6F31C6 der DE-10 Standard-Boards verwendet. Folgende Ressourcen werden für den \texttt{Coarse Alignment}-Block benötigt.
Diese sind in der nachstehenden Tabelle \ref{tab:resources_coarse} zu sehen.

\begin{table}[H]
\centering
\begin{tabular}{|c|c|}
\hline
\textbf{Ressource} & \textbf{Menge} \\ \hline
Register & 180 \\ \hline
ALMs & 168 \\ \hline
Memory Bits & 15360 \\ \hline
DSP-Blöcke & 2 \\ \hline
Pin (theoretisch) & 72 \\ \hline
\end{tabular}
\caption{Tabelle mit benötigten Ressourcen.}
\label{tab:resources_coarse}
\end{table}

Die Timing-Analyse hat ergeben, dass der \texttt{Coarse Alignment}-Block im 1100mV-85C-Model mit einer maximalen Taktfrequenz von 111,68 MHz betrieben werden kann. Als Systemtakt sind 100 MHz vorgesehen.

\subsection{Cyclic Prefix Removal}

\subsection{FFT Wrapper}

\subsection{Fine Alignment}

\subsection{Demodulation}

%\begin{thebibliography}{999}

%\bibitem [1]{gpioHelp1} Raspberry Pi GPIO via the Shell, \url{https://luketopia.net/2013/07/28/raspberry-pi-gpio-via-the-shell/}, 21. Dezember 2017 - 19:18 Uhr


%\end{thebibliography}

\end{document}
