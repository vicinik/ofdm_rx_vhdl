\documentclass[12pt,a4paper]{scrartcl}

\usepackage[a4paper, left=2cm, right=2cm, bottom=1cm, top=1cm, includeheadfoot]{geometry}
\usepackage[ngerman]{babel}
%\usepackage[utf8]{inputenc} % comment this if you uncomment utf8x
\usepackage[utf8x]{inputenc} % uncomment this if there are problems with 'ä', 'ü', 'ö'
\usepackage{ucs}
\usepackage[usenames,dvipsnames]{xcolor}
\usepackage[fleqn]{amsmath}
\usepackage{amsfonts}
\usepackage{amssymb}
\usepackage{color}
\usepackage{listings}
\usepackage{hyperref}
\usepackage{amsfonts}
\usepackage{scrpage2}
\usepackage{graphicx}
\usepackage{scrpage2} \pagestyle{scrheadings}
\usepackage[absolute]{textpos}
\usepackage{pdfpages}
\usepackage{alltt}
\usepackage{float}
\usepackage{array}
\usepackage{lastpage}
\usepackage{titling}

% command to format a collum in a table centered
\newcolumntype{C}[1]{>{\centering\arraybackslash}m{#1}}

\definecolor{mygray}{rgb}{0.95,0.95,0.95}
\lstset{language=[Visual]Basic, morekeywords={param, local}}

\lstset{
   literate={ö}{{\"o}}1
           {ä}{{\"a}}1
           {ü}{{\"u}}1
           {ß}{{\ss}}1
           {é}{{\'e}}1,
   inputencoding=ansinew,
   extendedchars=true,
   basicstyle=\scriptsize\ttfamily,
   numberstyle=\scriptsize,
   breaklines=true,
   tabsize=2,
   numbersep=5pt
}

\lstdefinestyle{ccode}{
   language=C,
   keywordstyle=\color{blue}\bfseries,
   stringstyle=\color{BrickRed}\ttfamily,
   commentstyle=\color{OliveGreen}\ttfamily,
   showspaces=false,
   showstringspaces=false,
   showtabs=false,
   numbers=left,
   frame=single
}
\lstdefinestyle{cplusplus}{
   language=C++,
   keywordstyle=\color{blue}\bfseries,
   stringstyle=\color{BrickRed}\ttfamily,
   commentstyle=\color{OliveGreen}\ttfamily,
   showspaces=false,
   showstringspaces=false,
   showtabs=false,
   numbers=left,
   frame=single
}
\lstdefinestyle{csharp}{
   language=[sharp]c,
   keywordstyle=\color{blue}\bfseries,
   stringstyle=\color{BrickRed}\ttfamily,
   commentstyle=\color{OliveGreen}\ttfamily,
   showspaces=false,
   showstringspaces=false,
   showtabs=false,
   numbers=none,
   frame=single
}
\lstdefinestyle{vhdlcode}{
   language=VHDL,
   keywordstyle=\color{blue}\bfseries,
   stringstyle=\color{BrickRed}\ttfamily,
   commentstyle=\color{OliveGreen}\ttfamily,
   showspaces=false,
   showstringspaces=false,
   showtabs=false,
   numbers=left,
   frame=single
}
\lstdefinestyle{customoutput}{
   backgroundcolor=\color{mygray},
   numbers=none,
   showspaces=false,
   showtabs=false
}

\newcommand{\includeccode}[1]{\lstinputlisting[style=ccode]{#1}}
\newcommand{\includecpluspluscode}[1]{\lstinputlisting[style=cplusplus]{#1}}
\newcommand{\includecsharpcode}[1]{\lstinputlisting[style=csharp]{#1}}
\newcommand{\includevhdlcode}[1]{\lstinputlisting[style=vhdlcode]{#1}}
\newcommand{\includetextfile}[1]{\lstinputlisting[style=customoutput]{#1}}

%pictures
\newcommand{\includepicture}[4]{
\begin{figure}[H]
\centering
\includegraphics[scale=#2]{#1}
\caption{\textit{#4}}
\label{fig:#3}
\end{figure}
}

%~~~~~~~~~~~~~~~~~~~~~~~~~~~~~~~~~~~~~~~~~~~~
% Some constants for header, footer and tile page
%~~~~~~~~~~~~~~~~~~~~~~~~~~~~~~~~~~~~~~~~~~~~

% Format der Kopf-und Fußzeile
\pagestyle{scrheadings}
\clearscrheadfoot

\ihead{\emph{FH Hagenberg / ESD}}
\chead{\emph{On Chip Signal Processing}}
\ohead{\emph{RTL Implementierung RX}}

\ofoot{\emph{\thepage \ / \ \pageref{LastPage}}}

\begin{document}  % Beginn des Textes

\pretitle{%
  \begin{center}
  \LARGE
  \includegraphics[scale=0.1]{../img/logo}\\[\bigskipamount]
}
\posttitle{\end{center}}

\title{On Chip Signal Processing - RTL Implementierung RX}
\subtitle{Sommersemester 2018}
\author{Daniel Ferrari \\ Nikolaus Haminger \\ Florian Kitzer \\ Matthäus Kücher \\ Alexander Traxler}

% Insert name and etc. into pdf
%\begin{textblock*}{60mm}(45mm,58mm)
%\firstname \ \lastname , \sfirstname \ \slastname
%\end{textblock*}

\maketitle % Titelseite erstellen
%\includepdf[pages=-]{../spv_ue05.pdf}
\tableofcontents

\newpage

\section{Übersicht}
\subsection{Spezifikation}
In diesem Projekt sollte eine OFDM-RX-Architektur in VHDL implementiert werden. Ein grobes Blockdiagramm dieser RX-Komponente war vorgegeben und ist in Abbildung \ref{fig:big_picture} sichtbar.

\includepicture{../img/big_picture.png}{0.45}{big_picture}{Blockschaltbild der gesamten RX-Kette.}

\subsection{Implementierung}
Die Implementierung wurde weitgehend nach dem Vorbild des spezifizierten Blockschaltbildes vollzogen. Allerdings wird der Offset und das Delay in der Entity \emph{Coarse Alignment} gehandhabt, weshalb die beiden Leitungen \texttt{rx\_data\_delay} und \texttt{rx\_data\_offset} zwischen \emph{Interpolation} und \emph{Coarse Alignment} eliminiert wurden.

\subsection{Toplevel-Simulation}
Zur Verifikation der gesamten RX-Kette wurde eine Toplevel-Simulation erstellt. Für den Vergleich der Ergebnisse der Hardware-Kette wurde ein Golden Model in Python implementiert. Diese Implementierung berechnet die Input-Signale für die Hardware-Simulation (\texttt{rx\_data\_i}), speichert diese in CSV-Dateien ab und simuliert die gesamte RX-Kette in einer Hardware-ähnlichen Implementierung. Die Ergebnisse des Golden Models werden ebenfalls abgespeichert und am Ende der Hardware-Simulation mit eben jenen Ergebnissen verglichen. Außerdem werden \emph{BER} und \emph{EVM} berechnet sowie Konstellationdiagramme jeweils für beide Simulationen erstellt.

\vspace{10pt}
\noindent
\textbf{Achtung}: Matlab wird nicht benötigt, da das gesamte Golden Model sowie die Verifikation in Python implementiert wurde (mit Fixed-Point-Repräsentation und Taylor-Approximation der Interpolation). Die Gründe für diese Maßnahme sind die bessere Performance von Python, die einfachere Interaktion aus der Modelsim-Simulation und die bessere Strukturierungsmöglichkeit des Codes.

\subsubsection{Anleitung zur Toplevel-Simulation}
Eine ausführliche Anleitung befinden sich in der \texttt{README.md} des Repositories. Hier kurz noch einmal zusammengefasst:

\vspace{5pt}
\noindent
\textbf{Anforderungen}:
\begin{itemize}
\item Modelsim oder Questasim
\item Python3 (https://www.python.org/downloads/)
\item numpy und matplotlib (\texttt{pip install numpy matplotlib})
\end{itemize}

\vspace{5pt}
\noindent
\textbf{Ausführung}:\\
Um die Simulation zu starten, einfach \texttt{start\_rx\_simulation.bat <cmd|gui|clean>} oder\\ \texttt{start\_rx\_simulation.sh <cmd|gui|clean>} im Root-Folder des Repositories ausführen.

\subsubsection{Simulations-Output}
Nachfolgend ist der Output einer Simulation von 3 unterschiedlichen RX-Sequenzen zu sehen.
\includetextfile{sim_output.txt}

\subsubsection{Konstellationsdiagramm}
Abbildung \ref{fig:plot0} zeigt die Konstellationsdiagramme einer Symbol-Sequenz für die Golden-Model- und die Hardware-Simulation. Die Plots sind nach einer Simulation im Ordner\\ \texttt{src/grpRx/unitTopLevel/sim} zu finden.
\includepicture{../img/scatter_plot0.png}{0.65}{plot0}{Konstellationsdiagramme der Golden-Model- und Hardware-Simulationen}

\section{Komponenten}

\subsection{Interpolation}

\subsection{Coarse Alignment}

\subsection{Cyclic Prefix Removal}

\subsection{FFT Wrapper}

\subsection{Fine Alignment}

\subsection{Demodulation}

%\begin{thebibliography}{999}

%\bibitem [1]{gpioHelp1} Raspberry Pi GPIO via the Shell, \url{https://luketopia.net/2013/07/28/raspberry-pi-gpio-via-the-shell/}, 21. Dezember 2017 - 19:18 Uhr


%\end{thebibliography}

\end{document}
